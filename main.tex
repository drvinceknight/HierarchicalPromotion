\documentclass{article}
\usepackage[margin=2.5cm, includefoot, footskip=30pt]{geometry}
\pagestyle{plain}
\setlength{\parindent}{0em}
\setlength{\parskip}{1em}
\renewcommand{\baselinestretch}{1}

\usepackage{amsmath}
\usepackage{amsfonts}

\newcommand{\R}{\mathbb{R}}

\title{The effect of homophily in hierarchical systems}

\renewcommand{\S}{\mathcal{S}}

\begin{document}

\maketitle


\section{Mathematical formulation of base model}

\begin{itemize}
    \item Given a hierarchical system with \(K\) levels.
    \item Level \(0\leq k < K\) has capacity \(C_k\)
    \item The first level (\(k=0\)) has the most capacity and capacity is
        monotonically decreasing: \(C_0 \geq C_1 \geq \dots \geq C_{k-2} \geq C_{k-1} = 1\).
    \item There are 2 types of agents: \(j\in\{0, 1\}\).
\end{itemize}

Consider a state space \(\S\):

\begin{equation}\label{eqn:state_space}
    \S = \left\{s \in \mathbb{Z} ^ {K \times 2} _ {\geq 0}\left|
        \begin{array}{l}
        s_{i0} + s_{i1} \leq C_i\text{ for all }0\leq i \leq K - 1\\
        s_{K-1} = (1, 0)\\
        \sum_{i=0}^{K - 1}s_{i0} + s_{i1} \in\left\{\sum_{i=0}^{K - 1}C_i, \sum_{i=0}^{K - 1}C_i - 1\right\}\\

        \end{array}
                \right.\right\}
\end{equation}
%TODO comment on top row
Where \(s_{ij}\) denotes the number of individuals of type \(j\) at level \(i\).

For example,

\begin{itemize}
    \item Let \(K = 3\)
    \item Let \(C = (4, 3, 1)\)
\end{itemize}

Then:

\[
    s = \begin{pmatrix}
        3 & 1 \\
        2 & 1 \\
        1 & 0 \\
    \end{pmatrix}
\]

corresponds to a system with 3 agents of first type and 1 of second type at the
first level, 2 of first type and 1 of second type at the second level and 1 of
each type at the 3rd level.

The constraints on \(\S\) ensure that either all positions are filled or a
single position is available. Thus at any stage either all spots are full and
someone will retire or there will be a spot available and someone will be
hired/promoted.

The size of the state space is then given by:

\begin{equation}
    |S| = \prod_{i=0}^{K - 2}\left(2C_i + 1\right)
\end{equation}

Given two elements \(s^{(1)}, s^{(2)}\in \S\) the transition rates are given by:

\begin{equation}\label{eqn:transition_rates}
    Q_{s_1, s_2} =
        \begin{cases}
            \mu_{ij},& \text{ if }s^{(2)} - s^{(1)} = -e_{ij}
            \text{ and }s^{(1)}_{i0} + s^{(1)}_{i1} = C_i\text{ for all }i\\
           \text{max}(rs_{ij} + s_{i, \bar j}, 1),& \text{ if }s^{(2)} - s^{(1)} = e_{ij} - e_{i-1, j}
                                     \text{ and }s^{(1)}_{i0} + s^{(1)}_{i1} < C_{i}
                                     \text{ and }i > 0\\
           \lambda_{j},&\text{ if }s^{(2)} - s^{(1)} = e_{0j}
                        \text{ and }s^{(1)}_{00} + s^{(1)}_{01} = C_{0} - 1\\
        \end{cases}
\end{equation}

Where:

\begin{itemize}
    \item \(\mu_{ij}\) is the retirement rate of agents of type \(j\) at level
        \(i\).
    \item \(r > 1\) is a constant that reflects the homophily effect.
    \item \(\lambda_j\) is the hiring rate of individuals of type \(j\).
\end{itemize}

% TODO Add some examples.

\section{Mathematical formulation of model with competence}

The base model described in the previous section does not take into account the
competence of the individuals. Let's consider the above example of:

\begin{itemize}
    \item \(K = 3\) and \(C = (4, 3, 1)\)
\end{itemize}

A possible state \(s\) for the above configuration when competence is taken into
account is:

\[
    s = \begin{pmatrix}
        (1, 0.1) & (1, 0.2) & (1, 0.3) & (0, 0.5) \\
        (1, 0.2) & (1, 0.3) & (0, 0.8) \\
        (1, 0.7) \\
    \end{pmatrix}
\]

Note that this is a possible state for the configurations of \(K\) and \(C\) but
not the only one. The competence of each individual can be any real number between
\([0, 1]\), thus there are an infinite number of possible states.

The set of possible states, denoted as \(S\), is given by,

\begin{equation}\label{eqn:state_space}
    \S = \left\{
        s \in (\{0, 1\}, \R^{[0,1]}) ^ {K \times C}  \left|
        \begin{array}{l}
         s_{i,j} = (l_{i, j}, c_{i, j}) \quad \forall i \in [0, K -1] \quad \forall j \in [C_i - 1, C_i] \\
        \text{ where } l_{i, j} \in \{0, 1\} \text{ and } c_{i, j} \in \R^{[0,1]} \\
        |s_{k - 1}| = 1  \\
        l_{k -1, 1} = 0
    \end{array}
    \right.\right\}
\end{equation}

where \(s_{i, j}\) is a tuple of \((l_{i, j}, c_{i, j})\) where \(l_{i, j}\) is the
type of individual \(j\) at level \(i\) and \(c_{i, j}\) is the individual's
competence.

Promotion in the competence model assumes that individuals of the same type give a random
bonus to the competence of individuals like them.
Thus for a state \(s\) where a promotion is happening at level \(i\), the promotion
probability of an individual \(j\) at level \(i - 1\) is given by:

\begin{equation}
    P(i, j) =
    \frac{
    \displaystyle \sum_{t = 1} ^ {C_{i -1} - 1} \text{max}((1 - |l_{i-i, t} - l_{i, j}|) \gamma_t, 1) \ c_{i, j}}
    {
    \displaystyle \sum_{\bar{j}=1}^{C_i} \sum_{t = 1} ^ {C_{i -1} - 1} \text{max}((1 - |l_{i-i, t} - l_{i, j}|) \gamma_t, 1) \ c_{i, \bar{j}}
    },
\end{equation}

where \(\gamma_{t} > 1\) is randomly sampled at each \(t\) from \([0, \Gamma]\).

Thus, the transition rates between two states are given by:

\begin{equation}\label{eqn:transition_rates}
        \begin{cases}
            \mu_{ij},& |s_i| = C_i\text{ for all }i\\
            P(i, j), & |s_i| < C_{i} \text{ and }i > 0\\
           \lambda_{(l, c)},& |s_{0}| = C_0 - 1\\
        \end{cases}
\end{equation}

Where:

\begin{itemize}
    \item \(\mu_{ij}\) is the retirement rate of the agent \(j\) at level
        \(i\).
    \item \(\lambda_{(l, c)}\) is the hiring rate of an individual \(l\) that has
    a competence \(c\).
\end{itemize}

The competence of the system is the total competence of the individuals in the
system regarding their type.

\begin{equation}\label{eqn:state_space}
    \sum_{i = 0} ^ {K - 1} \sum_{j=1} ^ {C_i} c_{i, j}
\end{equation}


\end{document}


