\documentclass{article}

\usepackage{amsmath}
\usepackage{amsfonts}

\title{The effect of homophily in hierarchical systems}

\renewcommand{\S}{\mathcal{S}}

\begin{document}

\maketitle


\section{Mathematical formulation of model}

\begin{itemize}
    \item Given a hierarchical system with \(K\) levels.
    \item Level \(0\leq k < K\) has capacity \(C_k\)
    \item The first level (\(k=0\)) has the most capacity and capacity is
        monotonically decreasing: \(C_0 > C_1 > \dots > C_{k-1}\).
    \item There are 2 types of agents: \(j\in\{0, 1\}\).
\end{itemize}

Consider a state space \(\S\):

\begin{equation}\label{eqn:state_space}
    \S = \left\{s \in \mathbb{Z} ^ {K \times 2} _ {\geq 0}\left|
        \begin{array}{l}
        s_{i0} + s_{i1} \leq C_i\text{ for all }0\leq i \leq K - 1\\
        s_{K-1,0} + s_{K-1,1} = C_{K - 1}\\
        \sum_{i=0}^{K - 1}s_{i0} + s_{i1} \in\left\{\sum_{i=0}^{K}C_i, \sum_{i=0}^{K}C_i - 1\right\}\\

        \end{array}
                \right.\right\}
\end{equation}

Where \(s_{ij}\) denotes the number of individuals of type \(j\) at level \(i\).

For example,

\begin{itemize}
    \item Let \(K = 3\)
    \item Let \(C = (5, 3, 2)\)
\end{itemize}

Then:

\[
    s = \begin{pmatrix}
        3 & 1 \\
        2 & 1 \\
        1 & 1 \\
    \end{pmatrix}
\]

corresponds to a system with 3 agents of first type and 1 of second type at the
first level, 2 of first type and 1 of second type at the second level and 1 of
each type at the 3rd level.

The constraints on \(\S\) ensure that either all positions are filled or a
single position is available. Thus at any stage either all spots are full and
someone will retire or there will be a spot available and someone will be
hired/promoted.

The size of the state space is then given by:

\begin{equation}
    |S| = (C_{K - 1} + 1)\cdot \prod_{i=0}^{K - 2}\left(2C_i + 1\right)
\end{equation}

Given two elements \(s^{(1)}, s^{(2)}\in \S\) the transition rates are given by:

\begin{equation}\label{eqn:transition_rates}
    Q_{s_1, s_2} =
        \begin{cases}
            \mu_{ij},& \text{ if }s^{(2)} - s^{(1)} = -e_{ij}
            \text{ and }s^{(1)}_{i0} + s^{(1)}_{i1} = C_i\text{ for all }i\\
           rS_{ij} + S_{i, \bar j},& \text{ if }s^{(2)} - s^{(1)} = e_{ij} - e_{i-1, j}
                                     \text{ and }s^{(1)}_{i0} + s^{(1)}_{i1} < C_{i}
                                     \text{ and }i > 0\\
           \lambda_{j},&\text{ if }s^{(2)} - s^{(1)} = e_{0j}
                        \text{ and }s^{(1)}_{00} + s^{(1)}_{01} = C_{0} - 1\\
        \end{cases}
\end{equation}

Where:

\begin{itemize}
    \item \(\mu_{ij}\) is the retirement rate of agents of type \(j\) at level
        \(i\).
    \item \(r > 1\) is a constant that reflects the homophily effect.
    \item \(\lambda_j\) is the hiring rate of individuals of type \(j\).
\end{itemize}

% TODO Add some examples.

\end{document}


